In questo documento vengono raccolti appunti del corso \textit{High Performance and Quantum Computing} tenuto nell'anno 2025-26 dal professor Cilardo presso l'università di Napoli Federico II, Dipartimento Ingegneria Elettrica e Tecnologie dell'Informazione.

\noindent \uppercase{è} assolutamente scoraggiata la vendita di questi appunti, che possono in qualsiasi momento essere scaricati gratuitamente da \href{https://github.com/h-tajato/hpqc_notes}{github}. Allo stessa repository è possibile in qualsiasi momento contruibuire seguendo le istruzioni esposte nel README.

\noindent Per la parte iniziale di ricapitolazione, sono stati utilizzati gli appunti raccolti durante il corso di Architettura e Progetto dei Calcolatori tenuto nell'anno 2025-26 dal professor Mazzocca presso l'università Federico II di Napoli (a cui è possibile accedere sia per consultare che per contribuire, nelle stesse modalità, su \href{https://github.com/Agda78/APC-appunti}{github}), i libri \textit{Computer Organization and Desing Patterns} (Hannessy-Patterson) e degli stessi autori \textit{Computer Architecture (a quantitative approach)}.

\noindent Una buona parte degli esempi a cui si fa riferimento, e di cui per motivi di spazio non verrà riportato l'intero svolgimento, è tratta dai lucidi messi a disposizione dal professore sul canale teams del corso. Ove necessario, sarà fatto esplicito riferimento a quelli. 
