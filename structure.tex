% %%%%%%%%%%%%%%%%%%%%%%%%%%%%%%%%%%%%%%%%%
% % LaTeX Template
% % Version 1.0
% %
% % This template originates from:
% % http://www.LaTeXTemplates.com
% %
% % Authors:
% % Rocco Lo Russo (https://github.com/roccolr)
% %%%%%%%%%%%%%%%%%%%%%%%%%%%%%%%%%%%%%%%%%


%----------------------------------------------------------------------------------------
%	PACKAGES AND OTHER DOCUMENT CONFIGURATIONS
%----------------------------------------------------------------------------------------

\usepackage{amsmath,amsfonts,stmaryrd,amssymb} % Math packages
\usepackage{graphicx}
\usepackage[ddmmyyyy]{datetime}
\usepackage{enumerate} % Custom item numbers for enumerations

\usepackage[ruled]{algorithm2e} % Algorithms

\usepackage[framemethod=tikz]{mdframed} % Allows defining custom boxed/framed environments

\usepackage{listings} % File listings, with syntax highlighting
\usepackage[hidelinks]{hyperref}
\usepackage{wrapfig}

\lstset{
  language=C,
  basicstyle=\ttfamily\small\color{black},
  numbers=left,
  numberstyle=\small\color{gray},
  stepnumber=1,
  numbersep=4pt,
  frame=single,
  rulecolor=\color{black},
  framesep=5pt,
  xleftmargin=20pt,
  framexleftmargin=15pt,
  breaklines=true,
  showstringspaces=false,
  keywordstyle=\color{purple}\bfseries,          % parole chiave (if, else, int, return...)
  commentstyle=\color{teal}\itshape,           % commenti
  stringstyle=\color{orange},                   % stringhe
  identifierstyle=\color{black},                % identificatori normali
  emph={size_t, uint32_t, uint64_t, TAS, RTE},            % tipi aggiuntivi
  emphstyle=\color{blue}\bfseries,            % stile tipi aggiuntivi
  morekeywords={uint8_t, uint16_t, uint32_t, uint64_t, int8_t, int16_t, int32_t, int64_t, bool, true, false}, % tipi e parole chiave extra
  morecomment=[l]{//},                           % commenti con //
  morecomment=[s]{/*}{*/},                       % commenti multilinea
}

\lstdefinelanguage{RISCAsm}{
  morekeywords={
    ADDQ, SUM, ANDI, MOVE,MOVEA, MOVEM, DC, DS, sra, 
    lw, sw, li, la, mv, 
    BEQ, BNE, JMP, CMP, jalr, ret, 
    NOP, lui, RTE, RTS, 
    ecall, ebreak, ORG, EQU, JSR, CLR, TAS 
  },
  sensitive=true,
  morecomment=[l]{\*},
  morestring=[b]",
  basicstyle=\ttfamily\small\color{black},
  keywordstyle=\color{purple}\bfseries,
  commentstyle=\color{teal}\itshape,
  stringstyle=\color{orange},
  numbers=left,
  numberstyle=\small\color{gray},
  stepnumber=1,
  numbersep=4pt,
  frame=single,
  rulecolor=\color{black},
  framesep=5pt,
  xleftmargin=20pt,
  framexleftmargin=15pt,
  showstringspaces=false,
  breaklines=true,
}

%----------------------------------------------------------------------------------------
%	DOCUMENT MARGINS
%----------------------------------------------------------------------------------------

\usepackage{geometry} % Required for adjusting page dimensions and margins

\geometry{
	paper=a4paper, % Paper size, change to letterpaper for US letter size
	top=2.5cm, % Top margin
	bottom=3cm, % Bottom margin
	left=2.5cm, % Left margin
	right=2.5cm, % Right margin
	headheight=14pt, % Header height
	footskip=1.5cm, % Space from the bottom margin to the baseline of the footer
	headsep=1.2cm, % Space from the top margin to the baseline of the header
	%showframe, % Uncomment to show how the type block is set on the page
}

%----------------------------------------------------------------------------------------
%	FONTS
%----------------------------------------------------------------------------------------

\usepackage[utf8]{inputenc} % Required for inputting international characters
\usepackage[T1]{fontenc} % Output font encoding for international characters

\usepackage{XCharter} % Use the XCharter fonts
\usepackage{enumitem}
\usepackage{xcolor}
\usepackage{tikz} 
\usetikzlibrary{positioning}
\setlist[itemize]{itemsep=0.2em, parsep=0pt}

% pseudocode environment %


%----------------------------------------------------------------------------------------
%	COMMAND LINE ENVIRONMENT
%----------------------------------------------------------------------------------------

% Usage:
% \begin{commandline}
%	\begin{verbatim}
%		$ ls
%		
%		Applications	Desktop	...
%	\end{verbatim}
% \end{commandline}

\mdfdefinestyle{commandline}{
	leftmargin=10pt,
	rightmargin=10pt,
	innerleftmargin=15pt,
	middlelinecolor=black!50!white,
	middlelinewidth=2pt,
	frametitlerule=false,
	backgroundcolor=black!5!white,
	frametitle={Command Line},
	frametitlefont={\normalfont\sffamily\color{white}\hspace{-1em}},
	frametitlebackgroundcolor=black!50!white,
	nobreak,
}

% Define a custom environment for command-line snapshots
\newenvironment{commandline}{
	\medskip
	\begin{mdframed}[style=commandline]
}{
	\end{mdframed}
	\medskip
}

%----------------------------------------------------------------------------------------
%	FILE CONTENTS ENVIRONMENT
%----------------------------------------------------------------------------------------

% Usage:
% \begin{file}[optional filename, defaults to "File"]
%	File contents, for example, with a listings environment
% \end{file}

\mdfdefinestyle{file}{
    innertopmargin=1.6\baselineskip,
    innerbottommargin=0.8\baselineskip,
    topline=false, bottomline=false,
    leftline=false, rightline=false,
    leftmargin=2cm,
    rightmargin=2cm,
    singleextra={
        \draw[fill=black!10!white](P)++(0,-1.2em)rectangle(P-|O);
        \node[anchor=north west] at(P-|O){\ttfamily\mdfilename};
        \def\l{3em}
        \draw(O-|P)++(-\l,0)--++(\l,\l)--(P)--(P-|O)--(O)--cycle;
        \draw(O-|P)++(-\l,0)--++(0,\l)--++(\l,0);
    },
}

% Comando per tab a 4 spazi all’interno dell’ambiente file
\newenvironment{file}[1][File]{%
    \medskip
    \newcommand{\mdfilename}{#1}
    % Rende il carattere tab attivo e lo definisce come 4 spazi
    \catcode`\^^I=\active
    \def^^I{\hspace*{4ex}}%
    \begin{mdframed}[style=file]
}{%
    \end{mdframed}
    \medskip
}

%----------------------------------------------------------------------------------------
%	NUMBERED QUESTIONS ENVIRONMENT
%----------------------------------------------------------------------------------------

% Usage:
% \begin{question}[optional title]
%	Question contents
% \end{question}

\mdfdefinestyle{question}{
	innertopmargin=1.2\baselineskip,
	innerbottommargin=0.8\baselineskip,
	roundcorner=5pt,
	nobreak,
	singleextra={%
		\draw(P-|O)node[xshift=1em,anchor=west,fill=white,draw,rounded corners=5pt]{%
		Question \theQuestion\questionTitle};
	},
}

\newcounter{Question} % Stores the current question number that gets iterated with each new question

% Define a custom environment for numbered questions
\newenvironment{question}[1][\unskip]{
	\bigskip
	\stepcounter{Question}
	\newcommand{\questionTitle}{~#1}
	\begin{mdframed}[style=question]
}{
	\end{mdframed}
	\medskip
}

%----------------------------------------------------------------------------------------
%	WARNING TEXT ENVIRONMENT
%----------------------------------------------------------------------------------------

% Usage:
% \begin{warn}[optional title, defaults to "Warning:"]
%	Contents
% \end{warn}

\mdfdefinestyle{warning}{
	topline=false, bottomline=false,
	leftline=false, rightline=false,
	nobreak,
	singleextra={%
		\draw(P-|O)++(-0.5em,0)node(tmp1){};
		\draw(P-|O)++(0.5em,0)node(tmp2){};
		\fill[black,rotate around={45:(P-|O)}](tmp1)rectangle(tmp2);
		\node at(P-|O){\color{white}\scriptsize\bf !};
		\draw[very thick](P-|O)++(0,-1em)--(O);%--(O-|P);
	}
}

% Define a custom environment for warning text
\newenvironment{warn}[1][Warning:]{ % Set the default warning to "Warning:"
	\medskip
	\begin{mdframed}[style=warning]
		\noindent{\textbf{#1}}
}{
	\end{mdframed}
}

%----------------------------------------------------------------------------------------
%	INFORMATION ENVIRONMENT
%----------------------------------------------------------------------------------------

% Usage:
% \begin{info}[optional title, defaults to "Info:"]
% 	contents
% 	\end{info}

\mdfdefinestyle{info}{%
	topline=false, bottomline=false,
	leftline=false, rightline=false,
	nobreak,
	singleextra={%
		\fill[black](P-|O)circle[radius=0.4em];
		\node at(P-|O){\color{white}\scriptsize\bf i};
		\draw[very thick](P-|O)++(0,-0.8em)--(O);%--(O-|P);
	}
}

% Define a custom environment for information
\newenvironment{info}[1][Info:]{ % Set the default title to "Info:"
	\medskip
	\begin{mdframed}[style=info]
		\noindent{\textbf{#1}}
}{
	\end{mdframed}
}


\hbadness=10000
\hfuzz=50pt